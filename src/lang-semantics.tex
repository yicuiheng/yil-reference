\subsection{\Yil の操作的意味}

\par \Yil の操作的意味を図~\ref{fig:lang:semantics} で定義する.

\begin{figure}[H]
  \begin{align*}
    \infer[\ERuleAppOp] {
      \Reduce{P}{\App{\mathrm{op}}{\seq{v}}}{\sem{\mathrm{op}}\paren{\seq{v}}}
    } {
      \arity{P}{\mathrm{op}} = |\seq{v}|
    } & \quad\quad
    \infer[\ERuleAppFunc] {
      \Reduce{P}{\App{f}{\seq{v}}}{\subst{\seq{v}}{\seq{x}}{e}}
    } {
      \Func{f}{\seq{x}}{e} \in P
      & |\seq{x}| = |\seq{v}|
    } \\\\
    \Reduce{P}{\If{\true}{e_1}{e_2}}{e_1} \quad \ERuleIfTrue & \quad\quad
    \Reduce{P}{\If{\false}{e_1}{e_2}}{e_2} \quad \ERuleIfFalse \\\\
    \infer[\ERuleLetCtx] {
      \Reduce{P}{\Let{x}{e_1}{e_2}}{\Let{x}{e'_1}{e_2}}
    } {
      \Reduce{P}{e_1}{e'_1}
    } & \quad\quad
    \Reduce{P}{\Let{x}{v}{e}}{\subst{v}{x}{e}} \quad \ERuleLetSubst
  \end{align*}

  \caption{\Yil の操作的意味}
  \label{fig:lang:semantics}
\end{figure}

\par $\Reduce{P}{e_1}{e_2}$ は プログラム $P$ における式 $e$ の1ステップの評価を表す.
有限ステップで $e_1$ が $e_2$ に簡約されることを $\MultiReduce{P}{e_1}{e_2}$ と書く.
\Yil の評価の例を~\ref{sec:lang:semantics:example} に示す.

