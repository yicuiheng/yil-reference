\subsection{\Yil の構文の形式的定義}

\par \Yil のコア言語は 図~\ref{fig:lang:syntax} で定義される.
\begin{figure}[h]
  \begin{align*}
    \text{(プログラム)} \; P & \seteq \Func{f}{\seq{x}}{e} \mid P, \; \Func{f}{\seq{x}}{e}                                         \\
    \text{(式)} \; e         & \seteq x \mid n \mid \mathrm{op} \mid \App{v_1 \;}{v_2} \mid \If{\; v \;}{\; e_1 \;}{\; e_2 \;} \mid \Let{\; x \;}{\; e_1 \;}{\; e_2 \;} \\
    \text{(値)} \; v         & \seteq x \mid n \mid \App{f}{\seq{v}} \where{|\seq{v}| < \arity{P}{f}}
  \end{align*}
  \caption{Yilコア言語の構文}
  \label{fig:lang:syntax}
\end{figure}

\par プログラム $P$ は,関数名が $f$,引数が $\seq{x}$, 本体が$e$ の再帰関数 $\Func{f}{\seq{x}}{e}$ の列である.
$x$, $y$ は式 $e$ の上のメタ変数であり,$f$ は関数変数の上のメタ変数である.
$\mathrm{op}$ は組み込み関数の変数の集合である.これには整数上の和 $+$, 差 $-$, 積 $\times$, 比較 $<$ が含まれる.
組み込み関数の関数適用は慣習に合わせた表記をする場合がある.つまり $x = y$ は $\App{\App{=}{x}}{y}$ と同等である. 
$\App{v_1}{v_2}$ は関数適用であり,$\If{\; v \;}{\; e_1 \;}{\; e_2 \;}$ は条件分岐を表す.$\If{\; v \;}{\; e_1 \;}{\; e_2 \;}$ は $v$ が $0$ の時に $e_1$ を評価し,$1$ の時に $e_2$ を評価する.
読みやすさのために $0$ を $\true$,$1$ を $\false$ と書くことがある.
組み込み関数変数 $\mathrm{not} \in \mathrm{op}$ は $\true$ を $\false$ に,$\false$ を $\true$ に写す関数である.
$\arity{P}{\cdot}$ はプログラム $P$ における関数の引数の数を表す. つまり $\Func{f}{\seq{x}}{e} \in P$ の時,
$\arity{P}{f} = |\seq{x}|$ であり $\arity{P}{+} = 2$ である.

\par \Yil 言語の実装と形式的定義とは異なる点がいくつかある.
例えば実装では if 式の条件部分に式を書けるが,形式的定義では型システムの都合上値に限定している.
つまり実装では $\If{\; x = 0 \;}{\; \true \;}{\; \false \;}$ と書けるが,
形式定義では $\Let{\; \code{cond}}{x = 0 \;}{\; \If{\; \code{cond} \;}{\; \true \;}{\; \false\;}}$ などと書く必要がある.
また,実装では式が出現しうるあらゆる場所に関数を書けるが,形式的定義ではトップレベルに限定している.
この違いは lambda lifting ~\cite{DBLP:conf/fpca/Johnsson85} というプログラム上のメタ操作によって無視できるので本質的な差にはならない.

\par 以降に使う補助関数を図~\ref{fig:lang:aux} に定義する.
$\fv{P}$, $\fv{e}$ はそれぞれプログラム $P$ もしくは式 $e$ の自由変数である.
さらに $\subst{e}{x}{P}$, $\subst{e_1}{x}{e_2}$ はそれぞれプログラム$P$ もしくは式 $e_2$ 中の自由変数 $x$ に式 $e_1$ を代入して得られたプログラムもしくは式を表す.
\begin{figure}[H]
  \begin{align*}
    \subst{e_1}{x}{\paren{\Func{f}{\seq{x}}{e_2}}} \; &\defeq \; \Func{f}{\seq{x}}{e} &\where{x = f \text{または} x \in \seq{x}} \\
    \subst{e_1}{x}{\paren{\Func{f}{\seq{x}}{e_2}}} \; &\defeq \; \Func{f}{\seq{x}}{\subst{e_1}{x}{\paren{e_2}}} &\where{x \not\equiv f \text{かつ} x \notin \seq{x}} \\
    \subst{e_1}{x}{\paren{P, \Func{f}{\seq{x}}{e_2}}} \; &\defeq \; \subst{e_1}{x}\paren{{P}}, \subst{e_1}{x}\paren{{\Func{f}{\seq{x}}{e_2}}} \\
    \subst{e_1}{x_1}{x_2} \; &\defeq \; x_2 &\where{x_1 \not\equiv x_2} \\
    \subst{e}{x}{x} \; &\defeq \; e \\
    \subst{e}{x}{n} \; &\defeq \; n \\
    \subst{e}{x}{\mathrm{op}} \; &\defeq \; \mathrm{op} \\
    \subst{e}{x}{\paren{\App{v_1 \;}{v_2}}} \; &\defeq \; \App{\paren{\subst{e}{x}{v_1}}}{\paren{\subst{e}{x}{v_2}}} \\
    \subst{e_1}{x}{\paren{\If{\; v \;}{\; e_2 \;}{\; e_3 \;}}} \; &\defeq \; \If{\; \subst{e_1}{x}{v} \;}{\; \subst{e_1}{x}{e_2} \;}{\; \subst{e_1}{x}{e_3}} \\
    \subst{e_1}{x}{\paren{\Let{\; x}{e_2 \;}{\; e_3}}} \; &\defeq \; \Let{\; x}{\subst{e_1}{x}{e_2\; }}{\; e_3} \\
    \subst{e_1}{x_1}{\paren{\Let{\; x_2}{e_2 \;}{\; e_3}}} \; &\defeq \; \Let{\; x_2}{\subst{e_1}{x_1}{e_2} \;}{\; \subst{e_1}{x_1}{e_3}} &\where{x_1 \not\equiv x_2} \\\\
    \fv{\Func{f}{\seq{x}}{e}} \; &\defeq \; \fv{e} \setminus \paren{\brace{f} \cup \brace{\seq{x}}} \\
    \fv{P, \Func{f}{\seq{x}}{e}} \; &\defeq \; \fv{P} \cup \fv{\Func{f}{\seq{x}}{e}} \\
    \fv{x} \; &\defeq \l \brace{x} \\
    \fv{n} \; &\defeq \; \emptyset \\
    \fv{\mathrm{op}} \; &\defeq \; \emptyset \\
    \fv{\App{v_1 \;}{v_2}} \; &\defeq \fv{v_1} \cup \fv{v_2} \\
    \fv{\If{\; v \;}{\; e_1 \;}{\; e_2}} \; &\defeq \; \fv{v} \cup \fv{e_1} \cup \fv{e_2} \\
    \fv{\Let{\; x}{e_1 \;}{\; e_2}} \; &\defeq \; \fv{e_1} \cup \paren{\fv{e_2} \setminus \brace{x}}
  \end{align*}
  \caption{補助関数}
  \label{fig:lang:aux}
\end{figure}

\NOTE{
  将来的にタプルや代数的データ構造およびパターンマッチも導入予定だが,今のところは単純のため非関数の式は整数しかないような言語を考える.
  そのような高度なデータ構造は ~\cite{DBLP:conf/popl/OngR11} や ~\cite{DBLP:conf/ppdp/UnnoK09} にあるような拡張によって無理なく扱えるようになると考えている.
}